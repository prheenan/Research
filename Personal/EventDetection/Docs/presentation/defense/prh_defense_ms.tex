% $Header$

% for options rationale, see:
% en.wikibooks.org/wiki/LaTeX/Colors#The_68_standard_colors_known_to_dvips
\documentclass[usenames,dvipsnames]{beamer}

% This file is a solution template for:

% - Giving a talk on some subject.
% - The talk is between 15min and 45min long.
% - Style is ornate.



% Copyright 2004 by Till Tantau <tantau@users.sourceforge.net>.
%
% In principle, this file can be redistributed and/or modified under
% the terms of the GNU Public License, version 2.
%
% However, this file is supposed to be a template to be modified
% for your own needs. For this reason, if you use this file as a
% template and not specifically distribute it as part of a another
% package/program, I grant the extra permission to freely copy and
% modify this file as you see fit and even to delete this copyright
% notice. 


\mode<presentation>
{
  \usetheme{Warsaw}
  % or ...

  \setbeamercovered{transparent}
  % or whatever (possibly just delete it)
}


\usepackage[english]{babel}
% or whatever

\usepackage[latin1]{inputenc}
% or whatever

\usepackage{times}
\usepackage[T1]{fontenc}
% Or whatever. Note that the encoding and the font should match. If T1
% does not look nice, try deleting the line with the fontenc.


\title[] % (optional, use only with long paper titles)
{Detecting Molecular Unbinding Events in Force Spectroscopy Data via a Bayesian Algorithm}

\subtitle
{} % (optional)

\author[] % (optional, use only with lots of authors)
{P.~Heenan\inst{1}}
% - Use the \inst{?} command only if the authors have different
%   affiliation.

\institute[University of Colorado at Boulder] % (optional, but mostly needed)
{
  \inst{1}%
  Department of Computer Science\\
  University of Colorado at Boulder
}
% - Use the \inst command only if there are several affiliations.
% - Keep it simple, no one is interested in your street address.

\date[M.S. Thesis Defense] % (optional)
{April 11 / M.S. Thesis Defense}

\subject{Talks}
% This is only inserted into the PDF information catalog. Can be left
% out. 

% styling, see:
% github.com/josephwright/beamer/blob/master/examples/a-conference-talk/beamerexample-conference-talk.tex
\usetheme{Darmstadt}
\usefonttheme[onlylarge]{structurebold}
\setbeamerfont*{frametitle}{size=\normalsize,series=\bfseries}
\setbeamertemplate{navigation symbols}{}
% remove footline, get more space. See:
% tex.stackexchange.com/questions/223200/how-to-completely-remove-footer-in-beamer
\setbeamertemplate{footline}[page number]
% remove the top subsection bar
% tex.stackexchange.com/questions/89748/beamer-remove-subsection-bar-in-header
\useoutertheme[subsection=false]{miniframes}
% If you have a file called "university-logo-filename.xxx", where xxx
% is a graphic format that can be processed by latex or pdflatex,
% resp., then you can add a logo as follows:

% \pgfdeclareimage[height=0.5cm]{university-logo}{university-logo-filename}
% \logo{\pgfuseimage{university-logo}}

% for drawing rectangles on parts of the plot 
% tex.stackexchange.com/questions/9559/drawing-on-an-image-with-tikz
\usepackage{tikz}
% set the itemize colors, subcolors, etc
% colors: 
% tex.stackexchange.com/questions/84958/changing-the-colour-of-the-text-with-itemize
\setbeamercolor{itemize/enumerate body}{fg=black}
\setbeamercolor{itemize/enumerate subbody}{fg=MidnightBlue}
\setbeamercolor{itemize/enumerate subsubbody}{fg=RedOrange}
% need bracket
\usepackage{braket}

% Delete this, if you do not want the table of contents to pop up at
% the beginning of each subsection:
\AtBeginSubsection[]
{
  \begin{frame}<beamer>{Outline}
    \tableofcontents[currentsection,currentsubsection]
  \end{frame}
}


% If you wish to uncover everything in a step-wise fashion, uncomment
% the following command: 

%\beamerdefaultoverlayspecification{<+->}

\graphicspath{{../../paper/Figures/Finals/},{../../paper/Figures/Finals_Presenation/}}
\newcommand{\DefineTerm}[1]{\underline{#1}}
\newcommand{\ErrorPct}[0]{P$_{85}$}

\begin{document}

\begin{frame}
  \titlepage
\end{frame}

\begin{frame}{Outline}
  \tableofcontents[pausesections]
  % You might wish to add the option [pausesections]
\end{frame}


% Since this a solution template for a generic talk, very little can
% be said about how it should be structured. However, the talk length
% of between 15min and 45min and the theme suggest that you stick to
% the following rules:  

% - Exactly two or three sections (other than the summary).
% - At *most* three subsections per section.
% - Talk about 30s to 2min per frame. So there should be between about
%   15 and 30 frames, all told.

\section{Introduction to Force Spectroscopy}

\subsection[Physical Background]{Physical Background}


\begin{frame}{Applying Forces by Atomic Force Microscopy (AFM)}

\begin{columns}[T]
\begin{column}{0.6\textwidth}
% Your text here
  \begin{itemize}
  \item
    Tip pulls surface-bound sample.
    \begin{itemize}
    \item
      No attachment
    \item<2->
      Single attachment
    \item<3->
      Multiple attachments
    \end{itemize}
  \invisible<1-3>{\item
    \DefineTerm{Rupture} ($\textcolor{OliveGreen}{\blacktriangledown}$): molecule unbinds}
  \only<5->{
  \item Linear fit $\rightarrow$ rupture properties 
    \begin{itemize}
    \item<6->
      \DefineTerm{Loading rate}: slope of fit
    \item<7->
      \DefineTerm{Rupture force}: fit value where data last above fit
    \end{itemize}
 }%
  \end{itemize}
\end{column}

% rupture image column 
\begin{column}{0.45\textwidth}  %%<--- here

\only<5->{ 
  \begin{figure}[ht]
    \centering
      \begin{tikzpicture}
% see: tex.stackexchange.com/questions/9559/drawing-on-an-image-with-tikz
% and: tex.stackexchange.com/questions/82530/how-to-draw-a-filled-rectangle-without-a-border-using-tikz
    \node[anchor=south west,inner sep=0] (image) at (0,0) {\includegraphics[width=0.9\textwidth]{ruptures.pdf}};
    \begin{scope}[x={(image.south east)},y={(image.north west)}]
        \only<5> \fill [white] (0.0,0.0) rectangle (1.0,0.65);%
        \only<6> \fill [white] (0.0,0.0) rectangle (1.0,0.33);%
        \only<7> \fill [white] (0.0,0.0) rectangle (0.0,0.0);%
    \end{scope}
\end{tikzpicture}
  \end{figure}
}%
\end{column}
\end{columns}



\invisible<5->{ 
  \begin{figure}[ht]
    \centering
      \begin{tikzpicture}
% see: tex.stackexchange.com/questions/9559/drawing-on-an-image-with-tikz
% and: tex.stackexchange.com/questions/82530/how-to-draw-a-filled-rectangle-without-a-border-using-tikz
    \node[anchor=south west,inner sep=0] (image) at (0,0) {\includegraphics[width=0.9\textwidth]{cartoon.pdf}};
    \begin{scope}[x={(image.south east)},y={(image.north west)}]
        \only<1> \fill [white] (0.4,0) rectangle (1.0,1.0);%
        \only<2> \fill [white] (0.7,0) rectangle (1.0,1.0);%
        \only<3> \fill [white] (1.0,0) rectangle (1.0,1.0);%
    \end{scope}
\end{tikzpicture}
  \end{figure}
}%
\end{frame}


\subsection{Detection of Rupture Events}

\begin{frame}{Past Work}


\begin{columns}[T]
\begin{column}{0.99\textwidth}
% Your text here
  \begin{itemize}
  \item
    Wavelet transforms
  \item<2->
    Contour length transform (Domain-Specific)
  \item<3->
    Classification
 \end{itemize}
\end{column}

\begin{column}{0.01\textwidth}
\end{column}
\end{columns}


\begin{overlayarea}{\textwidth}{\textheight}
  \begin{tikzpicture}
% see: tex.stackexchange.com/questions/9559/drawing-on-an-image-with-tikz
% and: tex.stackexchange.com/questions/82530/how-to-draw-a-filled-rectangle-without-a-border-using-tikz
    \node<1>[anchor=south west,inner sep=0] (image) at (0,0) {\includegraphics[width=0.8\textwidth]{bentez_2017_edit.png}};
    \node<2>[anchor=south west,inner sep=0] (image) at (0,0) {\includegraphics[width=0.8\textwidth]{kuhn_2005.png}};
    \node<3>[anchor=south west,inner sep=0] (image) at (0,0) {\includegraphics[width=0.8\textwidth]{Andreopoulos_2011_edit}};
\end{tikzpicture}
\end{overlayarea}

\end{frame}



\begin{frame}{Other Scientific Domains}
\end{frame}


\section{Methods}

\subsection{Sample Preparation and Data Acquisition}

\begin{frame}{Verifying sample purity}

\begin{columns}[T]
\begin{column}{0.6\textwidth}
% Your text here
  \begin{itemize}
  \item DNA purified by Agaorse gel 
  \invisible<-1>{\item DNA imaged by AFM}
  \end{itemize}
\end{column}
\begin{column}{0.4\textwidth}
\end{column}
\end{columns}


\begin{overlayarea}{\textwidth}{\textheight}
  \begin{tikzpicture}
% see: tex.stackexchange.com/questions/9559/drawing-on-an-image-with-tikz
% and: tex.stackexchange.com/questions/82530/how-to-draw-a-filled-rectangle-without-a-border-using-tikz
    \node<1-2>[anchor=south west,inner sep=0] (image) at (0,0) {\includegraphics[width=\textwidth]{prep.pdf}};
    \begin{scope}[x={(image.south east)},y={(image.north west)}]
        \only<-1> \fill [white] (0.4,0.0) rectangle (1.0,1.0);%
        \only<-2> \fill [white] (1.0,0.0) rectangle (1.0,1.0);%
    \end{scope}
\end{tikzpicture}
\end{overlayarea}

\end{frame}

\subsection{Algorithm Design, Theory, and Baselines}

\begin{frame}{Make Titles Informative.}


\end{frame}



\section{Algorithm Results}

\subsection{Timing}

\begin{frame}{Algorithm runtimes are linear}

\begin{columns}[T]
\begin{column}{0.99\textwidth}
% column about BCC
  \begin{itemize}
  \item Algorithm's runtime are linear with curve size
  \invisible<-1>{\item FEATHER has significantly better coefficients}
  \begin{itemize}
    \invisible<-1>{\item Minutes to analyze `realistic' dataset (hundreds of curves)}
  \end{itemize}
  \end{itemize}

\end{column}

\begin{column}{0.01\textwidth}
\end{column}

\end{columns}

\begin{overlayarea}{\textwidth}{\textheight}
  \begin{tikzpicture}
% see: tex.stackexchange.com/questions/9559/drawing-on-an-image-with-tikz
% and: tex.stackexchange.com/questions/82530/how-to-draw-a-filled-rectangle-without-a-border-using-tikz
    \node<1-2>[anchor=south west,inner sep=0] (image) at (0,0) {\includegraphics[width=\textwidth]{timing.pdf}};
    \begin{scope}[x={(image.south east)},y={(image.north west)}]
        \only<-1> \fill [white] (0.51,0.0) rectangle (1.0,1.0);%
        \only<-2> \fill [white] (1.0,0.0) rectangle (1.0,1.0);%
    \end{scope}
\end{tikzpicture}
\end{overlayarea}

\end{frame}

\subsection{Performance}

\begin{frame}{Distance Metric: Error Percentile \ErrorPct{} }

\begin{columns}[T]
\begin{column}{0.99\textwidth}
% column about BCC
  \begin{itemize}
  \item \DefineTerm{\textcolor{OliveGreen}{d$_{\mathrm{t}\rightarrow\mathrm{p}}$}}: distance (meters) from true to predicted%
  \invisible<-1>{\item \DefineTerm{\textcolor{Blue}{d$_{\mathrm{p}\rightarrow\mathrm{t}}$}}: distance (meters) from predicted to true}%
  \end{itemize}
\end{column}

\begin{column}{0.01\textwidth}
\end{column}

\end{columns}



\begin{overlayarea}{\textwidth}{\textheight}
  \begin{tikzpicture}
% see: tex.stackexchange.com/questions/9559/drawing-on-an-image-with-tikz
% and: tex.stackexchange.com/questions/82530/how-to-draw-a-filled-rectangle-without-a-border-using-tikz
    \node<1-3>[anchor=south west,inner sep=0] (image) at (0,0) {\includegraphics[width=\textwidth]{FEATHER_distances.pdf}};
    \begin{scope}[x={(image.south east)},y={(image.north west)}]
        \only<-2> \fill [white] (0.5,0.0) rectangle (1.0,1.0);%
        \only<-3> \fill [white] (1.0,0.0) rectangle (1.0,1.0);%
    \end{scope}
    \node<4>[anchor=south west,inner sep=0] (image) at (0,0) {\includegraphics[width=\textwidth]{FEATHER_dist.pdf}};
    \node<5>[anchor=south west,inner sep=0] (image) at (0,0) {\includegraphics[width=\textwidth]{OpenFovea_dist.pdf}};
    \node<6>[anchor=south west,inner sep=0] (image) at (0,0) {\includegraphics[width=\textwidth]{ScientificPython_dist.pdf}};
\end{tikzpicture}
\end{overlayarea}

\end{frame}

\begin{frame}{Rupture Metric: Bhattacharya Coefficient's Complement (BCC)}

\begin{columns}[T]
\begin{column}{0.99\textwidth}
% column about BCC
  \begin{itemize}
  \item \DefineTerm{BCC}: disagreement between distributions (X,Y)
    \begin{itemize}
    \item
      Mathematically: $1-\braket{X^{\frac{1}{2}}|Y^{\frac{1}{2}}}$
    \item 
      Informally: probability mass difference
    \invisible<-4>{\item Open Fovea, Scientific Python, FEATHER: }
    \end{itemize}
  \end{itemize}
\end{column}

\begin{column}{0.01\textwidth}
\end{column}
\end{columns}

%
% % bcc description
%
\begin{overlayarea}{\textwidth}{\textheight}
\begin{tikzpicture}
% see: tex.stackexchange.com/questions/9559/drawing-on-an-image-with-tikz
% and: tex.stackexchange.com/questions/82530/how-to-draw-a-filled-rectangle-without-a-border-using-tikz
    \node<1-5>[anchor=south west,inner sep=0] (image) at (0,0) {\includegraphics[width=\textwidth]{bcc.pdf}};
    \begin{scope}[x={(image.south east)},y={(image.north west)}]
        \only<-1> \fill [white] (0.28,0.0) rectangle (1.0,1.0);%
        \only<-2> \fill [white] (0.53,0.0) rectangle (1.0,1.0);%
        \only<-3> \fill [white] (0.76,0.0) rectangle (1.0,1.0);%
        \only<-4> \fill [white] (1.0 ,0.0) rectangle (1.0,1.0);%
    \end{scope}
    \node<5>[anchor=south west,inner sep=0] (image) at (0,0) {\includegraphics[width=\textwidth]{FEATHER_rupture.pdf}};
    \node<6>[anchor=south west,inner sep=0] (image) at (0,0) {\includegraphics[width=\textwidth]{OpenFovea_rupture.pdf}};
    \node<7>[anchor=south west,inner sep=0] (image) at (0,0) {\includegraphics[width=\textwidth]{ScientificPython_rupture.pdf}};
\end{tikzpicture}
\end{overlayarea}

\end{frame}


\section*{Summary}

\begin{frame}{Summary}

  % Keep the summary *very short*.
  \begin{itemize}
  \item
    The \alert{first main message} of your talk in one or two lines.
  \item
    The \alert{second main message} of your talk in one or two lines.
  \item
    Perhaps a \alert{third message}, but not more than that.
  \end{itemize}
  
  % The following outlook is optional.
  \vskip0pt plus.5fill
  \begin{itemize}
  \item
    Outlook
    \begin{itemize}
    \item
      Something you haven't solved.
    \item
      Something else you haven't solved.
    \end{itemize}
  \end{itemize}
\end{frame}


\end{document}
