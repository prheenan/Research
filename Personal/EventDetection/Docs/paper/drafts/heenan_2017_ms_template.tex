% ****** Start of file aipsamp.tex ******
%
%   This file is part of the AIP files in the AIP distribution for REVTeX 4.
%   Version 4.1 of REVTeX, October 2009
%
%   Copyright (c) 2009 American Institute of Physics.
%
%   See the AIP README file for restrictions and more information.
%
% TeX'ing this file requires that you have AMS-LaTeX 2.0 installed
% as well as the rest of the prerequisites for REVTeX 4.1
%
% It also requires running BibTeX. The commands are as follows:
%
%  1)  latex  aipsamp
%  2)  bibtex aipsamp
%  3)  latex  aipsamp
%  4)  latex  aipsamp
%
% Use this file as a source of example code for your aip document.
% Use the file aiptemplate.tex as a template for your document.
\documentclass[%
  aip,12pt,tightenlines,
  amsthm,
 amsmath,amssymb
]{article}

\usepackage{graphicx}% Include figure files
\usepackage{dcolumn}% Align table columns on decimal point
\usepackage{bm}% bold math
%\usepackage[mathlines]{lineno}% Enable numbering of text and display math
%\linenumbers\relax % Commence numbering lines
\usepackage{txfonts}
\usepackage{natbib}
\usepackage{tabularx}
% enable clicking on citations
\usepackage{hyperref}
\usepackage{titlesec}
% for degree
\usepackage{gensymb}
% use times new roman
% http://tex.stackexchange.com/questions/153168/how-to-set-document-font-to-times-new-roman-by-command
\usepackage{times}
% 1 inch margins
% http://kb.mit.edu/confluence/pages/viewpage.action?pageId=3907057
\usepackage[margin=1in]{geometry}
% reducing heading madness
% http://tex.stackexchange.com/questions/53338/reducing-spacing-after-headings
\newcommand{\captionspacing}[0]{4pt plus 2pt minus 2pt}
\titlespacing\section{0pt}{\captionspacing}{\captionspacing}
\titlespacing\subsection{0pt}{\captionspacing}{\captionspacing}
\titlespacing\subsubsection{0pt}{\captionspacing}{\captionspacing}



\newcommand{\ndiffn}[3]{\frac{\partial^{#3}#1}{\partial^{#3}#2}}
\newcommand{\diffn}[2]{\ndiffn{#1}{#2}{}}
\newcommand{\eqs}[1]{
\begin{align*} 
\begin{split}
#1
\end{split}					
\end{align*}}

\newcommand{\pFig}[3]{
\begin{figure}[h!]
\centering
\pcaption{#2}
\boxed{\includegraphics[width=0.75\textwidth]{#1}}
\label{figure:#3}
\end{figure}}



\newcommand{\eqlab}[2]{
\begin{equation}
\label{equation:#2}
\begin{split}
#1
\end{split}
\end{equation}
}


\newcommand{\brac}[1]
{
  \Big[ #1 \Big]
}

\newcommand{\tab}[2]
{
\begin{tabular}[h!]{#1}
\hline
#2
\end{tabular}
}

\newcommand{\e}[0]{\\ \hline}

\newcommand{\pMat}[1]{
 \left[ \begin{array}
#1 \end{array} \right]
}

\newcommand{\code}[1]{
  \begin{lstlisting}[language=Python]
#1
\end{lstlisting}
}

% for making specific references.
\newcommand{\tRef}[1]{Table \ref{table:#1}}
\newcommand{\fRef}[1]{Figure \ref{figure:#1}}
\newcommand{\sRef}[1]{Section \ref{section:#1}}
\newcommand{\eRef}[1]{Equation \ref{equation:#1}}

\newcommand{\fLabel}[1]{\label{figure:#1}}
\newcommand{\tLabel}[1]{\label{table:#1}}
\newcommand{\sLabel}[1]{\label{section:#1}}
% command for my captionc
\newcommand{\pcaption}[1]{\caption{\noindent#1}}
% command for first paragraph
\newcommand{\firstp}[0]{\noindent}
% command for ending a paragraph
\newcommand{\pl}[0]{\vspace{6pt}}
\newcommand{\pEndF}[0]{ \\ }
\newcommand{\pStartF}[0]{ }

% single-spaced captions. see:
% tex.stackexchange.com/questions/153334/change-the-baselinestretch-line-spacing-only-for-figure-captions
\usepackage{graphicx,kantlipsum,setspace}
\usepackage{caption}
\captionsetup[table]{font={stretch=1}}    
\captionsetup[figure]{font={stretch=1}}


\newcommand{\pLit}[4]{
\cite{#1}
\begin{adjustwidth}{2.5em}{0pt}
\textbf{Title}: \citetitle{#1} \\
\textbf{Keywords}: #4 \\
\textbf{Big Picture}: #2 \\
\textbf{Summary}: #3 
\end{adjustwidth}
}

% supply: for 'citing' where we got a reagent or thing
%    Args:
%        #1: name of company we got the thing from
%        #2: product number for the thing
\newcommand{\supply}[2]{(#1 #2)}
% singlemol: for writing single molecular force spectroscopy
\newcommand{\singlemol}[0]{SMFS}
% degreeC: for abbreviating degrees celsius
\newcommand{\degreeC}[0]{\degree{}C}

\newcommand{\pSkip}[0]{0pt}
\makeatletter
\g@addto@macro\normalsize{%
  \setlength\abovedisplayskip{\pSkip}
  \setlength\belowdisplayskip{\pSkip}
  \setlength\abovedisplayshortskip{\pSkip}
  \setlength\belowdisplayshortskip{\pSkip}
}

\linespread{2.0}

\captionsetup[figure]{format=hang,justification=raggedright,indent=-22pt}
\captionsetup[table]{format=hang,justification=raggedright,indent=-31pt}
% make it so there arent terrible spaces everywehre
% see :
% tex.stackexchange.com/questions/36423/random-unwanted-space-between-paragraphs
\raggedbottom

\begin{document}

% single spaced
\singlespacing

\title{Mapping Subsurface Elemental Composition Using X-ray Fluorescence}
\date{\today}
\author{Patrick Heenan \\  University of Colorado at Boulder }

\maketitle

\clearpage


\begin{abstract}

\firstp Non-destructive methods for identifying elemental makeup provide valuable insight for researchers investigating object composition. X-ray fluorescence (XRF) has emerged as an \emph{in situ} method to obtain the elemental makeup of objects, including sub-surface composition. For example, XRF can reconstruct “hidden” paintings beneath a surface image, resolve disputed claims of authorship, and recover text from partially effaced religious documents. XRF traditionally uses synchrotron radiation as a light source and raster scans the artifact of interest. The pixels from the scan are combined into a coherent picture after days of imaging. Recent work focuses on creating two-dimensional images of the artifact's sub-surface elemental composition in hours or days by digitizing photon counts onto a charged coupled device camera. The number of photons in each energy bin for each pixel, constituting the scattered and fluorescent light from the object, are converted into element concentrations by comparing the photon intensities with a prior calibration and correcting for instrumental artifacts and background. This work describes the physical theory, optical instrumentation, and data analysis underpinning recent advances in spatial and temporal resolution in sub-surface reconstruction using XRF. 
\end{abstract}



%\pacs{32.30.Rj, 32.50.+d, 32.70.Fw, 32.70.Jz}% PACS, the Physics and Astronomy
                             % Classification Scheme.
%\keywords{Review, x-ray fluorescence, scattering, attenuation}%Use showkeys class option if keyword
                              %display desired
\maketitle

\doublespacing

\section{\sLabel{Intro}Introduction}

outline:\\
Cool examples \\
how it works  (XXX why surface bound is good?)\\
especially talk about dynamic force spectroscopy \\
why data processing is difficult \\
XXX anatomy of FEC
previous techniques\\

\firstp Atomic force microscopy (AFM) is a powerful tool for studying the mechanical properties of biological molecules.  AFM instruments can resolve sub-nanometer molecular structure such as the major and minor grooves of DNA \cite{ido_beyond_2013}, lattice structure of membrane-bound proteins \cite{muller_surface_1999}, and real-time motion of motor proteins\cite{ando_high-speed_2007}. As a force spectroscopy platform, AFM XXX. AFM is attractive for studies of biological systems since it operates with high resolution in a wide range of temperatures, solvents, and other environmental variables CITE() \pl

During an AFM experiment, an atomically sharp tip attached to a cantilever interacts with a sample(XXX figure). The interaction is measured by the displacement of the tip via deflection of the cantilever CITE(XXX). A calibrated tip can record interaction forces from piconewtons to nanonewtons CITE(XXX). \pl

In single molecular force spectroscopy (\singlemol{}) experiments, the AFM tip interacts with a molecule and stretches it to measure the relationship between force and extension over time. This experiment records the force-extension curve of the molecule (see figure XXX). Force-extension curves yield information such as kinetic information for dissociation of bonds (Cite XXX), protein-ligand bond strength (cite XXX), and the energy landscapes of proteins and nucleic acids (cite XXX). XXX insights \pl

The analysis and interpretability of \singlemol{} experiments is limited by the attachment of the tip to the molecule of interest. Even with a high-concentration sample, attachment rates between commerically-available AFM tips and most molecules are much less than 0.1\% (XXX cite, see Figure XXX). Attachment rates are improved by coating the AFM tip with a molecule exhibiting high binding affinity for the sample of interest (cite XXX), but the majority of the data taken is still uninterpretable and must be filtered out.\pl

Methods have been proposed to automate manually sorting \singlemol{} data. Automation removes the burden of tedious data filtering and improves scientific reproducibility. XXX LIST examples \pl

This paper describes a new method based on machine learning for detecting events in SMFS force-extension curves.  First, we describe the sample preparation of functionalized AFM cantilevers samples of single-stranded and double-stranded DNA. We then mannually annotate the unbinding events in the force-extension curves of the DNA at multiple loading rates, contour lengths, and molecular attachments per curve (XXX cite available online? cite table giving data information?). Finally, we discuss a new method for automatically identifying events in force-extension curves and highlight its improved performance to previously proposed methods. \pl


\section{\sLabel{Materials}Materials and Methods}

The paper uses site-specific chemistry to improve \singlemol{} data quality and acquisition rate. The sample polymer is functionalized with DBCO at one end to ensure a covelent bond with an azide-functionalized surface, and the polymer is functionalized with biotin at the opposite end to ensure a specific but reverseible bond with the streptavidin-coated tip. These two bonds ensure the tip-polymer bond is broken before the surface-polymer bond, preventing tip contamination. This section descibes how the desired samples, surfaces, and tips are created.

\subsection{\sLabel{Surface}Azide-functionalied surfaces}

\firstp Glass surfaces were smoothed using 3M Potassium Hydroxide etching as follows. 12mm diameter glass disks \supply{Ted Pella}{26023} were sonicated \supply{Branson}{2510} for 5 minutes in 250mL of 95\% ethanol \supply{Decon}{2801}, 250mL of ACS grade acetone \supply{Fisher}{A18-4}, and transferred to a 3M solution of Potassium Hydroxide \supply{Fisher}{P250-500} on a benchtop for 3 minutes. To removed residual KOH, the glass was serially diluted in two 1L beakers of 18.2M$\Omega$ deionized water \supply{Barnstead}{GenPure Pro}. The etched surfaces were dried with 99.8\% pure Nitrogen gas \supply{Airgas}{NI UHP-300} and stored at room temperature. \pl

To create azide-functionalized surfaces, etched glass was activated using 30 minutes exposure in a UV-ozone chamber \supply{Novascan}{PSDP Digital UV Ozone Sysrem} and incubated for three hours in a 70mL, 60\degreeC{} solution of 0.15mg/mL 600Dalton Silane-PEG-Azide \supply{Nanocs}{PG2-AZSL-600} dissolved in toluene \supply{Sigma}{179418-4L}. The surfaces were then rinsed serially in 250mL of toluene, isopropanol \supply{Fisher}{A416-4}, and deionized water. The azide-functionalized surfaces were dried with Nitrogen gas and stored in a dust-proof container at 4\degreeC{}. \pl 

\subsection{\sLabel{Sample}DNA Samples}

\firstp For the 1914bp double-stranded DNA, the 1607 forward-sense and 3520 reverse-sense primers (\tRef{Sequences}) for the M13mp18 plasmid \supply{New England BioLabs}{N4018S} were obtained from Integrated DNA Technologies  and used for polymerase-chain reaction \supply{Millipore}{71086-4} using 40 cycles on a thermocycler \supply{Bio-Rad}{T100}. The reverse-sense primer was modified to include four biotinylated thyimine bases, and the forward-sense primer was modified to include a 5-prime DBCO after a PEG-spacer. After the desired PCR product was separated by 2\% gel electrophoresis \supply{Sigma}{A9539-500}, the agarose was eluted out \supply{Bio-Rad}{732-6165}, the DNA solution was concentrated \supply{Millipore}{UFC501096}, purified \supply{Qiagen}{28106}, and stored at 4\degreeC{} in a solution of TE, 10mM Tris \supply{Fisher}{BP151-1} and 1mM EDTA \supply{Fisher}{S311-500} at pH 8.0. The typical recovery efficiency of this procedure was 25\%. \pl 

For the 68bp single-stranded DNA, the purified construct (\tRef{Sequences}) was obtained from IDT, aliquotted to 10$\muup$M in a solution of TE at pH 8.0, flash-frozen in liquid Nitrogen \supply{MVE Cyrogenics}{VL-160} and stored at -20\degreeC{}. Aliquots were thawed to room temperature on a metal block then kept at 4\degreeC{} as needed. \pl 

For DNA deposition, 20uL of the desired concentration of DNA (see section XXX) was mixed with 80uL of TE at pH 8.0 and despoited onto azide-functionalized glass (see section XXX) affixed with epoxy \supply{Devcon}{14250} to specimen disks \supply{Ted Pella}{16218} and incubated at 4C for 15 hours. The surfaces were rinsed by 7mL of TE pH 8.0 and 7mL of PBS with 1mM EDTA at pH 7.4, and stored at 4\degreeC{}. 

\subsection{\sLabel{Cantilevers}Streptravidin-functionalized cantilevers}

\firstp Functionalized cantilevers with nominal spring constants of 4pN/nm  were used for all experiments. Functionalization etches away gold to improve force stability (XXX cite) and covalently attaches streptatividin to the etched cantilevers to improve attachment to the biotinylated DNA. Commercially available cantilevers \supply{Asylum Research}{BL-RC-150VB} were serially rinsed for 30 seconds in 50mL of gold etchant \supply{Transene}{TFA}, 250mL of deionized water, 50mL of chromium etchant \supply{Transene}{1020}, 250mL of deionized water, and one final rinse in 250mL of deionized water. After drying (XXX -- solvent rinses?), the tips were activated using 30 minutes exposure in a UV-Ozone chamber and incubated for three hours in a 70mL, 60\degreeC{} solution of 0.15mg/mL 600Dalton Silane-PEG-Maleimide \supply{Nanocs}{PG2-MLSL-600} dissolved in toluene. The maleimide-functionalized tips were serially rinsed in 50mL of toluene, isopropopanol, water, immediately dried \supply{KimTech}{5511}, and immersed for three hours in a 0.4mg/mL solution of thiol-streptavidin \supply{Protein Mods}{SAVT} in PBS \supply{Millipore}{524650} at pH 6.75 with 1mM TCEP \supply{Thermo Scientific}{77720} at room temperature in a moisture-proof container. The tips were then tranferred to 4\degreeC{} for 15 hours. After the 4\degreeC{} incubation, to remove free streptavidin the tips were serially rinsed in two 10mL beakers of PBS at pH 7.4, submerged in a 20mL petri dish of PBS for 10 minutes. The tips were then stored in 50$\muup$L of PBS at 4C in plastic wafers \supply{Entegris}{H22-10/1-0615} until loaded into the atomic force microscope. \pl

\subsection{\sLabel{Surface}Atomic force microscopy}

All atomic force microcopy measurements were carried out using an Asylum Cypher \supply{Asylum Research}{Cypher ES}. The spring constant and sensitivity were determined using the equipartition theorem method (cite XXX). All measurements were carried out with a 2s pause at the surface (see XXX) at the loading rates noted in the text. To avoid possible non-uniform distributions of the sample in spaces, measurements were taken at points 1$\muup$m separated over a 30$\muup$m x 30$\muup$m grid. 

\subsection{\sLabel{Surface}Data annotation and analysis}

\section{\sLabel{Results}Results}

\section{\sLabel{Discussion}Discussion}


\firstp Since the discovery of x-rays in 1895 by Wilhelm Conrad R{\"o}ntgen, researchers leveraging x-rays have dramatically improved our understanding of the world, receiving multiple Nobel prizes along the way.\cite{santra_concepts_2009} Fundamental scientific discoveries in early x-ray studies include: illuminating crystal structure by Bragg's x-ray diffraction experiments in 1913,\cite{bragg_reflection_1913} revealing electron orbital state through inelastic x-ray scattering by Compton in 1923,\cite{compton_spectrum_1923} and Klein's generalizations of Compton scattering to higher-energy photons using quantum electrodynamics in 1928.\cite{klein_scattering_1928} In addition to serving as a probe for fundamental investigations of matter, x-rays also find applications in quantifying and identifying industrial pollutants;\cite{luo_determination_2012} imaging oncological and vascular tissues;\cite{butler_bio-medical_2008} and recovering, restoring, or establishing the provenance of works of artistic \cite{janssens_photon-based_2010} or historical \cite{bergmann_archimedes_2007} importance. All these applications are made possible by determining the elemental makeup of the object of interest. \pl

%\begin{figure}
%\includegraphics[width=3in]{../../Figures/Figure0.pdf}% Here is how to import EPS art
%\pcaption{\noindent\fLabel{Motivation}\pStartF An example subsurface reconstruction of a self-portrait, 200mm$\times$300mm, of Australian painter Sir Arthur Streeton (1867-1943) from \citet{howard_high-definition_2012} [a] The visible image of the painting is obscured by primer. [b] A log-scaled intensity image of the obscured painting's zinc content. \pEndF }
%\end{figure}

\subsubsection{\sLabel{Beer}XRF depends on wavelength and element}

\firstp XRF is characterized by absorption of a photon by an electron bound to an atom, followed by `relaxation' of the excited atom by emission of an x-ray photon.\cite{jitschin_progress_1990} Relaxation occurs by XRF when the excited electron is ejected and a higher-shell electron `decays' to fill the vacancy, emitting radiation as a result of the decay (\fRef{Basics}). The relaxation process can consist of a complicated series of excitations and decays,\cite{santra_concepts_2009} but the highest-energy and most-radiative initial excitation and decay is the primary focus of XRF experiments and this paper. Photon absorption and electron ejection happen at well-defined incident photon energies, called the \emph{edge energies}, which are a function of the atomic number of the absorbing atom and the ejected electron's shell.\cite{ueda_high-resolution_2003,de_groot_high-resolution_2001} \fRef{Basics} shows the edge energies and naming conventions associated with the highest-energy and most-dominant\cite{chantler_detailed_2000} absorption levels for the first 90 elements. The tightest-bound, 1s$_{\frac{1}{2}}$ electrons, known as the K-shell electrons, are the dominant contributors to XRF by absorption of the incident photons.\cite{chantler_detailed_2000}

\begin{table}
\begin{tabularx}{\textwidth}{ l | l  }
Forward primer & Dagttgttcctttctattctcactccgc \\ \e 
Reverse primer & BtcaataaTcggctgtctTtccttatcaTtc \\ \e 
68nt hairpin & Dttgagtcaacgtactgatcacgctggatcctat\\
& ttttaggatccagcgtgatcagtacgttgactcttB \\ \e 
\end{tabularx}
\pcaption{\tLabel{Sequences}Sequences for single-stranded DNA construct and for double-stranded DNA primers. The forward and reverse primers amplify a 647nm piece of DNA, from positions 1607 to 3520 on the M13mp18 plasmid, as discussed in the next. Unmodified DNA bases are lowercase. The uppercase letters `T',`B', and `D' respectively represent biotinylated Thymidine residues, a terminal Biotin separated from the sequence by triethyleneglycol spacer, and a terminal Dibenzocyclooctyl separated from the sequence by triethyleneglycol spacer.}
\end{table}

%\newpage


\section{References}


\section{Bibliography}

\bibliographystyle{unsrtnat}

\bibliography{0ms-citations} 

\section{Suporting Material}


\end{document}

%
% XXX difference wrt to solids. Auger, no space? \\

%

%
% ****** End of file aipsamp.tex ******
