\chapter{\sLabel{SampleDetails}Sample and cantilever preparation}

\section{\sLabel{Surface}Azide-functionalized surfaces}

Glass surfaces were etched using potassium hydroxide. \ValUnit{12}{mm} diameter glass disks \supply{Ted Pella}{26023} were sonicated \supply{Branson}{2510} for 5 minutes in \ValUnit{250}{mL} of ACS grade acetone \supply{Fisher}{A18-4}, 5 minutes in \ValUnit{250}{mL} of 95\% ethanol \supply{Decon}{2801}, and transferred to a solution of \ValUnit{80}{g} of potassium hydroxide \supply{Fisher}{P250-500} dissolved in  \ValUnit{170}{mL} of 95\% ethanol and \ValUnit{80}{mL} of deionized water \supply{Barnstead}{GenPure Pro} for 3 minutes. To removed residual KOH, the glass was serially diluted in two \ValUnit{1}{L} beakers of \ValUnit{18.2}{M$\Omega$} deionized water. The etched surfaces were dried with 99.8\% pure nitrogen gas \supply{Airgas}{NI UHP-300} and stored at room temperature in a dust-proof box. 

To create azide-functionalized surfaces, etched glass was activated using 30 minutes exposure in an UV-ozone chamber \supply{Novascan}{PSDP Digital UV Ozone System} and incubated for three hours in a \ValUnit{70}{mL}, 60\degreeC{} solution of \ValUnit{0.15}{mg/mL} \ValUnit{600}{Dalton} Silane-PEG-Azide \supply{Nanocs}{PG2-AZSL-600} dissolved in toluene \supply{Sigma}{179418-4L}. The surfaces were then mounted in a custom-made teflon-holder and rinsed serially in \ValUnit{250}{mL} of toluene, isopropanol \supply{Fisher}{A416-4}, and deionized water. The azide-functionalized surfaces were dried with nitrogen gas and stored in a dust-proof container at 4\degreeC{}.  


\begin{table}[htp]
\caption[DNA primer sequences]{\tLabel{Sequences}Sequences for single-stranded DNA construct and for double-stranded DNA primers. The forward and reverse primers amplify a \ValUnit{647}{nm} piece of DNA, from positions 1607 to 3520 on the M13mp18 plasmid, as discussed in the text. Unmodified DNA bases are lowercase. The uppercase letters `T',`B', and `D' respectively represent biotinylated Thymidine residues, a terminal Biotin separated from the sequence by triethyleneglycol spacer, and a terminal dibenzocyclooctyl (DBCO) separated from the sequence by triethyleneglycol spacer.}
\begin{tabularx}{\textwidth}{ l | l  }
\hline \hline
Forward primer & Dagttgttcctttctattctcactccgc \\ \e 
Reverse primer & BtcaataaTcggctgtctTtccttatcaTtc \\ \e 
\end{tabularx}
\end{table}

\section{\sLabel{Sample}DNA Samples}

For the \ValUnit{1914}{basepair (bp)} double-stranded DNA, the 1607 forward-sense and 3520 reverse-sense primers (\tRef{Sequences}) for the M13mp18 plasmid \supply{New England BioLabs}{N4018S} were obtained from Integrated DNA Technologies  and used for polymerase-chain reaction \supply{Millipore}{71086-4} using 40 cycles on a thermocycler \supply{Bio-Rad}{T100}. The reverse-sense primer was modified to include three biotinylated thymidine bases and a 5-prime biotin after a PEG spacer. The forward-sense primer was modified to include a 5-prime dibenzocyclooctyl (DBCO) after a PEG-spacer (see \tRef{Sequences}). After the PCR product was purified \supply{Qiagen}{28106} and the \ValUnit{1.9}{kbp} band selected by 2\% gel electrophoresis \supply{Sigma}{A9539-500} as shown in \fRef{Prep}, the agarose was eluted out \supply{Bio-Rad}{732-6165}, the DNA solution was concentrated \supply{Millipore}{UFC501096} and purified \supply{Qiagen}{28106}. The purified construct was stored at 4\degreeC{} in a solution of \ValUnit{10}{mM} Tris \supply{Fisher}{BP151-1} and \ValUnit{1}{mM} EDTA \supply{Fisher}{S311-500} at pH 8.0, referred to after as `TE'. The typical recovery efficiency of DNA purification after polymerase chain reaction was 25\%. 

The purity of the DNA was verified by depositing \ValUnit{20}{pmol} of DNA in imaging buffer \textemdash{} \ValUnit{3}{mM} Nickel \supply{Sigma}{654597}, \ValUnit{10}{mM} Hepes \supply{Sigma}{H4034} at pH 7 \textemdash{} onto freshly-cleaved, \ValUnit{10}{mm} diameter, V1 mica \supply{Ted Pella}{50} for 10 minutes, rinsing serially 5 times with \ValUnit{1}{mL} of deionized water and 5 times with \ValUnit{1}{mL} of imaging buffer. The sample was then imaged with the Cypher AFM using a cantilever with a nominal \ValUnit{2}{nm} radius, a spring constant of $\approx$300 $\frac{\text{pN}}{\text{nm}}$ \supply{Bruker}{SNL-10}, a line scan rate of \ValUnit{1}{Hz}, a \ValUnit{1}{\textmu{}m} scan size, and free amplitude of $\approx$\ValUnit{1}{nm} (\fRef{Prep}). 

For DNA deposition onto an azide surface, \ValUnit{20}{\textmu{}L} of the DNA at \ValUnit{40}{nM} was mixed with \ValUnit{80}{uL} of TE and deposited onto azide-functionalized glass (see \sRef{Surface}) affixed with 5 minute epoxy \supply{Devcon}{14250} to specimen disks \supply{Ted Pella}{16218} and incubated at 4\degreeC{} overnight. The surfaces were rinsed by \ValUnit{7}{mL} of TE at pH 8.0 and \ValUnit{7}{mL} of Phosphate Buffered Saline \supply{Millipore}{524650}, referred to after as PBS, with \ValUnit{1}{mM} EDTA at pH 7.4, and stored at 4\degreeC{}. 

\section{\sLabel{Cantilevers}Streptravidin-functionalized cantilevers}

Functionalized cantilevers with nominal spring constants of 4 $\frac{\text{pN}}{nm}$  were used for all experiments. The protocol below etches away gold to improve force stability\cite{sullan_atomic_2013} and covalently attaches streptatividin to the etched cantilevers to improve attachment to the biotinylated DNA. AFM cantilevers \supply{Asylum Research}{BL-RC-150VB} were serially rinsed for 30 seconds in \ValUnit{50}{mL} of gold etchant \supply{Transene}{TFA}, \ValUnit{250}{mL} of deionized water, \ValUnit{50}{mL} of chromium etchant \supply{Transene}{1020}, and \ValUnit{250}{mL} of deionized water. The tips were then serially rinsed for 30 seconds in \ValUnit{50}{mL} of deionized water, isopropanol, toluene, isopropanol, and deionized water again. After drying, the tips were activated via 30 minutes exposure in a UV-Ozone chamber and incubated for three hours in a \ValUnit{70}{mL}, 60\degreeC{} solution of \ValUnit{0.15}{mg/mL} \ValUnit{600}{Dalton} Silane-PEG-Maleimide \supply{Nanocs}{PG2-MLSL-600} dissolved in toluene. The maleimide-functionalized tips were serially rinsed in \ValUnit{50}{mL} of toluene, isopropanol, and water, immediately dried \supply{KimTech}{5511}, and immersed for three hours in a \ValUnit{0.2}{mg/mL} solution of thiol-streptavidin \supply{Protein Mods}{SAVT} in PBS at pH 6.75 with \ValUnit{1}{mM} TCEP \supply{Thermo Scientific}{77720} at room temperature in a moisture-proof container. Subsequently, the tips were transferred to 4\degreeC{} overnight. After the 4\degreeC{} incubation, to remove free streptavidin, the tips were serially rinsed in two \ValUnit{10}{mL} beakers of PBS at pH 7.4 and submerged in a \ValUnit{20}{mL} petri dish of PBS for 10 minutes. These functionalized tips were stored in \ValUnit{50}{\textmu{}L} of PBS at 4\degreeC{} in plastic wafers \supply{Entegris}{H22-10/11-0615} until loading into the atomic force microscope. 



\chapter{\sLabel{DesignDetails} Algorithm design}

This section details the event-detection algorithm. The following conventions are followed:

\begin{enumerate}
 \item All variables with uppercase letters are random variables.
 \item All variables with lowercase letters are either instances of the corresponding uppercase random variables (\textit{i.e.}, measurements) or pure functions.
 \item All variables with a hat (\textit{e.g.}, $\hat{\epsilon}_t$) are estimators of a variable.
\end{enumerate}

\section{Defining the no-event hypothesis}

\name{} defines an event as a discontinuity in piecewise-continuous time series data. In \singlemol{}, events occur when the force applied to a molecule exhibits a discontinuity as the molecule passes over an energy barrier. \name{} assumes that events take place on time scales much smaller than the response time of the probe. If this is not true, the data is assumed smoothed until this condition is reached. The algorithm models the data assuming no event is occurring at a given time `$t$' and finds where the probability of a measurement is low. Hereafter, the definition of an event and these assumptions will be referred to as the \emph{no-event hypothesis}. 

\section{Mathematical background for testing the no-event hypothesis}

Under the no-event hypothesis, the noise-dependent distribution of force `F$_t$' for a discrete series of forces sampled at time points `t' can be well-approximated by the sum of a smooth signal and a noise distribution (\fRef{FeatherExample}):

\eqs{ F_t = g_t + X(0,\sigma^2) }

where `g$_t$' is a smooth function with a continuous first derivative and `X' is a random variable with zero mean and variance $\sigma^2$. The closed form of the smooth signal and the noise distribution are assumed unknown and can vary from one \fec{} to the next. If `$g^{*}_t$' is a function with a continuous first derivative approximately equal to $g_t$ such that $\forall t,\epsilon_t\equiv|g^{*}_t-g_t|$ for real $\epsilon_t\ge 0$, then the error distribution $R_t$ is defined such that: 

\eqlab{ R_t \equiv F_t - g^{*}_t = \epsilon_t + X(0,\sigma) }{error}

where

\eqlab{ E[R_t^2] -E[R_t]^2 = [(g_t-g^{*}_t)^2 + E[X(0,\sigma)^2]] - (g_t-g^{*}_t)^2  = \sigma^2 }{feather-sigma}

and 

\eqlab{ |E[R_t]| = \epsilon_t \le E[|R_t|] }{feather-epsilon}

Under these assumptions, the probability `P' of measuring $r_t$ is bounded by Chebyshev's inequality:

\eqlab{ \boxed{P( |R_t-\epsilon_t| \ge |r_t-\epsilon_t| ) \le
 (\frac{\sigma}{|r_t-\epsilon_t|})^2 }}{feather-probability}


For \emph{any} noise distribution, \eRef{feather-probability} bounds the probability of a measurement under the no-event hypothesis, given the force approximation $g^{*}_t$ (which in turn yields the estimator error $\hat{\epsilon}$ and the noise $\hat{\sigma}$ by \eRef{feather-epsilon} and \eRef{feather-sigma}). A low probability at a given time implies the measurement is unlikely under the no-event hypothesis. 

\section{Accurate estimators for hypothesis testing}

To obtain an accurate estimator for $g_t$, the data must be smoothed. The approximation to the noiseless force $g^{*}_t$ is obtained by fitting a least-squares second-order basis spline\citePRH{dierckx_algorithm_1975} to the force versus time curve. The spline is second-order to ensure a continuous first derivative (\fRef{FeatherExample}), and the spline knots are spaced uniformly at intervals of  the user-specified $\tau$ (see \tRef{Parameters}). \fRef{FeatherExample} is a representative demonstration of the spline fitting. Determining  $g^{*}_t$ immediately gives $\hat{r}_t$ by  \eRef{error}.

Using $r_t$ as shown in \eRef{feather-epsilon} does not provide a strong signal in the presence of an event (see \fRef{FeatherExample}). In order to improve the method, $r_t$ was replaced by the distribution of windowed standard deviations `$\WindowStdev{}$'. $\WindowStdev{}$ is defined as the standard deviation of $r_t$ centered at t with a window of $[-\frac{\tau}{4},\frac{\tau}{4}]$. Using $\WindowStdev{}$ instead of $r_t$ provides a much stronger signal in the presence of an event (see \fRef{FeatherExample}).  

The noise variables $\sigma$ and $\epsilon$ are estimated from the distribution of standard deviations $\WindowStdev{}$ on the region of the approach curve where the AFM tip is not in contact with the surface. From this distribution, $\hat{\sigma}$ is set to the standard deviation of $\WindowStdev{}$, and $\hat{\epsilon}_t$ is approximated by the median. The median is used instead of the mean to remove the influence of possible false positive events in the approach. The removal of these pseudo-events is necessary to ensure accurate estimators for $\sigma$ and $\epsilon$, which are based on the no-event hypothesis. 

The quality of \name{}'s results are improved by multiplying the no-event probability, as discussed above, by the integral force, force derivative, and force differential Chebyshev probabilities. This is explicitly detailed in \fRef{Code}. The calculation of each of these probabilities is exactly the same as \eRef{feather-probability}, with the variables changed appropriately. Specifically, the relevant operation (integration, differentiation, or force difference) is applied to the approach, estimates for the operation-specific $\epsilon$ and $\sigma$ are obtained, yielding the operation-specific probability.


\begin{figure}[htp]
\caption[\name{} algorithmic pipeline]{\noindent\fLabel{FeatherExample} Demonstrating how \name{} works. \subref{A,B} The approach and retract forces versus time, with spline fits overlaid. \subref{C,D} The approach and retract r$_t$ versus time, with $\WindowStdev{}$ overlaid, demonstrating the signal-to-noise benefit of using $\WindowStdev{}$. \subref{E,F} The approach and retract $\WindowStdev{}$, with the estimates of $\epsilon$ and $\sigma$ from the approach overlaid. \subref{G} The probability of no event at each time has a sharp minima near the expected event location. For all subplots, the retract changes color at the location of a tagged event.  }
\centering
\includegraphics[width=\figwidth]{../Figures/Finals/algorithm.pdf}% Here is how to import EPS art
\end{figure}

\begin{figure}[htp]
\caption[\name{} psuedocode]{\noindent\fLabel{Code} The psuedocode of \name{}. \name{} takes in `approach' and `retract' as the relevant portions of the \fec{}; `threshold', which is a number between 0 and one defining the maximum probability; and `$\tau$' which is the even number of points between the spline knots.  }
  \begin{lstlisting}[language=Python]
def event_indices(approach,retract,threshold,$\tau$):
   g$^{*}$ = second-order spline fit to retract using $\tau$ for knots
   g$_a$ = second order spline fit to approach using $\tau$ for knots
   r = retract - g$^{*}$
   r$_a$ = approach - g$^{*}$
   s[i] = standard deviation of r[i-$\frac{\tau}{4}$:i+$\frac{\tau}{4}$]
   s$_a$[i] = standard deviation of r$_a$[i-$\frac{\tau}{4}$:i+$\frac{\tau}{4}$]
   $\epsilon$ = median of s$_a$
   $\sigma$ = standard deviation of s$_a$
   k = (s-$\epsilon$)/$\sigma$
   probability = mininimum(1,k$^{-2}$)
   # multiply by the integral probability 
   integrated$_{s}$[i] = integrate (s-$\epsilon$) from [i-$\frac{\tau}{4}$] to [i+$\frac{\tau}{4}$]
   set integrated$_{s}$ where positive to $\sigma$
   k$_{\text{integ}}$ = (integrated$_{s}$/$\sigma$)
   probability = probability * minimum(1,k$_{\text{integ}}^{-2}$)
   # multiply by the derivative probability
   $\epsilon_d$  = median of derivative of g$_a$
   $\sigma_d$ = standard deviation of derivative of g$_a$
   g$_{\text{deriv}}$ = derivative of $g^{*}$ with respect to time
   k$_{\text{deriv}}$ = (g$_{\text{deriv}}$ - $\epsilon_d$)/$\sigma_d$
   probability = probability * minimum(1,k$_{\text{deriv}}^{-2}$)
   # multiply by the differential probability 
   df[i] = g$^{*}$[i+$\frac{\tau}{2}$] - g$^{*}$[i-$\frac{\tau}{2}$]
   k$_{\text{df}}$ = (df - $\epsilon$)/$\sigma$
   probability = probability * minimum(1,k$_{\text{df}}^{-2}$)
   possible_events = indices where probability <= threshold
   # update the probability to remove false positives 
   surface_location = last point where appproach > 0
   set events containing surface_location to 0
   remove events where retract-df $\le$ zero
   remove events where df within (epsilon+sigma) of zero
   return possible_events
\end{lstlisting}
\end{figure}



\begin{table}[htp]
\caption[Data set statistical information]{\tLabel{statistics} For each loading rate v in the data set, this table lists the number of curves N$_{\mathrm{curves}}$; mean and standard deviation of curve sizes, $\mu_{\mathrm{Curve Size}}$ and $\sigma_{\mathrm{Curve Size}}$, in data points; the number of curves with `x' events N$_{\mathrm{e=x}}$ for x$\in\{1,2,3\}$; and the number of curves with greater than or equal to 4 events, $N_{\mathrm{e}\ge4}$. }
\begin{tabularx}{\textwidth}{ l | l | l | l |l |l|l|l }
\hline \hline
v (nm/s) & N$_\mathrm{curves}$ & $\mu_{\mathrm{Curve Size}}$ & $\sigma_{\mathrm{Curve Size}}$ & N$_{\mathrm{e}= 1}$ & N$_{\mathrm{e}= 2}$ & N$_{\mathrm{e}= 3}$ & N$_{\mathrm{e}\ge4}$  \\ \hline
100 & 200 & 667000 & 47000 & 159 & 33 & 5 & 3  \\ \hline
500 & 200 & 200000 & 1000 & 140 & 40 & 8 & 12  \\ \hline
1000 & 200 & 117000 & 7000 & 174 & 25 & 1 & 0  \\ \hline
\end{tabularx}
\end{table}



\begin{figure}[htpb]
\caption[Algorithm performance]{\noindent\fLabel{Performance} Performance of \name{} compared to the baseline algorithms. This figure is the concatenation of \fRef{PerformanceFEATHER}, \fRef{PerformanceFovea}, and \fRef{PerformanceScipy}. The range of the loading rate and rupture force scatter plots and histograms are set to encompass the predicted data range all algorithms.}
\centering
\includegraphics[width=\figwidth]{../Figures/Finals/landscape.pdf}% Here is how to import EPS art
\end{figure}



\begin{figure}[htp]
\caption[Performance of \name{} on larger data set]{\noindent\fLabel{LargeDataset} This figure is identical to \fRef{Performance}, except it details the performance of only \name{} on the full data set listed in \tRef{statistics}, instead of merely the highest loading rate.   }
\centering
\includegraphics[width=\figwidth]{../Figures/Finals/FEATHER_full.pdf}% Here is how to import EPS art
\end{figure}

\begin{figure}[htp]
\caption[Verification of sample purity]{\noindent\fLabel{Prep} Purity of the sample preparation. \subref{A} 2\% electrophoretic agarose gel, showing a major band just below 2kbp, as expected. \subref{B} A false-color AFM image of the DNA bound to mica.  }
\centering
\includegraphics[width=\figwidth]{../Figures/Finals/prep.pdf}% Here is how to import EPS art
\end{figure}


\begin{figure}[htp]
\caption[Cross validation of algorithms and optimal parameters]{\noindent\fLabel{Tuning} Details of tuning experiments. \subref{A} The \BccLong{} metric versus tuning parameter for \name{}. \subref{B} As (A), but for the OpenFovea method. \subref{C} As (A), but for the Scientific Python method. In all subplots, only tuning points which return at least one prediction are plotted. }
\centering
\includegraphics[width=\figwidth]{../Figures/Finals/tuning.pdf}% Here is how to import EPS art
\end{figure}



\begin{figure}
\caption[Algorithmic runtime versus loading rate]{\noindent\fLabel{Timing_Details} The runtime of the three methods versus loading rate and number of points. \subref{A} \name{}'s total runtime versus number of curves analyzed for the curve sizes, N, listed. \subref{B} The runtime per curve versus number of curve points, N. The runtimes at each N are obtained by the slope of the relevant line in (A).  \subref{C,D} As (A,B), but for the Open Fovea method. \subref{E,F} As (A,B), but for the Scientific Python method.  }
\centering
\includegraphics[width=\figwidth]{../Figures/Finals/supplemental.pdf}% Here is how to import EPS art
\end{figure}



\end{document}

%

%
% ****** End of file aipsamp.tex ******
