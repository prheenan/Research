% ****** Start of file aipsamp.tex ******
%
%   This file is part of the AIP files in the AIP distribution for REVTeX 4.
%   Version 4.1 of REVTeX, October 2009
%
%   Copyright (c) 2009 American Institute of Physics.
%
%   See the AIP README file for restrictions and more information.
%
% TeX'ing this file requires that you have AMS-LaTeX 2.0 installed
% as well as the rest of the prerequisites for REVTeX 4.1
%
% It also requires running BibTeX. The commands are as follows:
%
%  1)  latex  aipsamp
%  2)  bibtex aipsamp
%  3)  latex  aipsamp
%  4)  latex  aipsamp
%
% Use this file as a source of example code for your aip document.
% Use the file aiptemplate.tex as a template for your document.
\documentclass[%
  aip,12pt,tightenlines,
  amsthm,
 amsmath,amssymb,
%preprint,%
 reprint,%
%author-year,%
%author-numerical,%
]{revtex4-1}
\usepackage{graphicx}% Include figure files
\usepackage{dcolumn}% Align table columns on decimal point
\usepackage{bm}% bold math
%\usepackage[mathlines]{lineno}% Enable numbering of text and display math
%\linenumbers\relax % Commence numbering lines
\usepackage{txfonts}
\usepackage{natbib}
% enable clicking on citations
\usepackage{hyperref}
\usepackage{titlesec}

% use times new roman
% http://tex.stackexchange.com/questions/153168/how-to-set-document-font-to-times-new-roman-by-command
\usepackage{times}
% 1 inch margins
% http://kb.mit.edu/confluence/pages/viewpage.action?pageId=3907057
\usepackage[margin=1in]{geometry}
% reducing heading madness
% http://tex.stackexchange.com/questions/53338/reducing-spacing-after-headings
\newcommand{\captionspacing}[0]{4pt plus 2pt minus 2pt}
\titlespacing\section{0pt}{\captionspacing}{\captionspacing}
\titlespacing\subsection{0pt}{\captionspacing}{\captionspacing}
\titlespacing\subsubsection{0pt}{\captionspacing}{\captionspacing}



\newcommand{\ndiffn}[3]{\frac{\partial^{#3}#1}{\partial^{#3}#2}}
\newcommand{\diffn}[2]{\ndiffn{#1}{#2}{}}
\newcommand{\eqs}[1]{
\begin{align*} 
\begin{split}
#1
\end{split}					
\end{align*}}

\newcommand{\pFig}[3]{
\begin{figure}[h!]
\centering
\pcaption{#2}
\boxed{\includegraphics[width=0.75\textwidth]{#1}}
\label{figure:#3}
\end{figure}}



\newcommand{\eqlab}[2]{
\begin{equation}
\label{equation:#2}
\begin{split}
#1
\end{split}
\end{equation}
}


\newcommand{\brac}[1]
{
  \Big[ #1 \Big]
}

\newcommand{\tab}[2]
{
\begin{tabular}[h!]{#1}
\hline
#2
\end{tabular}
}

\newcommand{\e}[0]{\\ \hline}

\newcommand{\pMat}[1]{
 \left[ \begin{array}
#1 \end{array} \right]
}

\newcommand{\code}[1]{
  \begin{lstlisting}[language=Python]
#1
\end{lstlisting}
}

% for making specific references.
\newcommand{\tRef}[1]{Table \ref{table:#1}}
\newcommand{\fRef}[1]{Figure \ref{figure:#1}}
\newcommand{\sRef}[1]{Section \ref{section:#1}}
\newcommand{\eRef}[1]{Equation \ref{equation:#1}}

\newcommand{\fLabel}[1]{\label{figure:#1}}
\newcommand{\tLabel}[1]{\label{table:#1}}
\newcommand{\sLabel}[1]{\label{section:#1}}
% command for my captionc
\newcommand{\pcaption}[1]{\caption{\noindent#1}}
% command for first paragraph
\newcommand{\firstp}[0]{\noindent}
% command for ending a paragraph
\newcommand{\pl}[0]{\vspace{6pt}}
\newcommand{\pEndF}[0]{ \\ }
\newcommand{\pStartF}[0]{ }

% single-spaced captions. see:
% tex.stackexchange.com/questions/153334/change-the-baselinestretch-line-spacing-only-for-figure-captions
\usepackage{graphicx,kantlipsum,setspace}
\usepackage{caption}
\captionsetup[table]{font={stretch=2}}    
\captionsetup[figure]{font={stretch=2}}


\newcommand{\pLit}[4]{
\cite{#1}
\begin{adjustwidth}{2.5em}{0pt}
\textbf{Title}: \citetitle{#1} \\
\textbf{Keywords}: #4 \\
\textbf{Big Picture}: #2 \\
\textbf{Summary}: #3 
\end{adjustwidth}
}

\newcommand{\pSkip}[0]{0pt}
\makeatletter
\g@addto@macro\normalsize{%
  \setlength\abovedisplayskip{\pSkip}
  \setlength\belowdisplayskip{\pSkip}
  \setlength\abovedisplayshortskip{\pSkip}
  \setlength\belowdisplayshortskip{\pSkip}
}

\linespread{2.0}

\captionsetup[figure]{format=hang,justification=raggedright,indent=-22pt}
\captionsetup[table]{format=hang,justification=raggedright,indent=-31pt}
% make it so there arent terrible spaces everywehre
% see :
% tex.stackexchange.com/questions/36423/random-unwanted-space-between-paragraphs
\raggedbottom

\begin{document}

\preprint{AIP/123-QED}



\title[CU Boulder, Comps II, 2016]{Mapping Subsurface Elemental Composition Using X-ray Fluorescence}% Force line breaks with \\

\author{Patrick Heenan }
\affiliation{ 
University of Colorado at Boulder, Department of Physics%\\This line break forced with \textbackslash\textbackslash
}%

\date{\today}% It is always \today, today,
             %  but any date may be explicitly specified

\newpage

\begin{abstract}

\firstp Non-destructive methods for identifying elemental makeup provide valuable insight for researchers investigating object composition. X-ray fluorescence (XRF) has emerged as an \emph{in situ} method to obtain the elemental makeup of objects, including sub-surface composition. For example, XRF can reconstruct “hidden” paintings beneath a surface image, resolve disputed claims of authorship, and recover text from partially effaced religious documents. XRF traditionally uses synchrotron radiation as a light source and raster scans the artifact of interest. The pixels from the scan are combined into a coherent picture after days of imaging. Recent work focuses on creating two-dimensional images of the artifact's sub-surface elemental composition in hours or days by digitizing photon counts onto a charged coupled device camera. The number of photons in each energy bin for each pixel, constituting the scattered and fluorescent light from the object, are converted into element concentrations by comparing the photon intensities with a prior calibration and correcting for instrumental artifacts and background. This work describes the physical theory, optical instrumentation, and data analysis underpinning recent advances in spatial and temporal resolution in sub-surface reconstruction using XRF. 
\end{abstract}




%\pacs{32.30.Rj, 32.50.+d, 32.70.Fw, 32.70.Jz}% PACS, the Physics and Astronomy
                             % Classification Scheme.
%\keywords{Review, x-ray fluorescence, scattering, attenuation}%Use showkeys class option if keyword
                              %display desired
\maketitle


\section{\sLabel{Intro}Introduction}

\section{\sLabel{Materials}Materials and Methods}

\section{\sLabel{Results}Results}

\section{\sLabel{Discussion}Discussion}


\firstp Since the discovery of x-rays in 1895 by Wilhelm Conrad R{\"o}ntgen, researchers leveraging x-rays have dramatically improved our understanding of the world, receiving multiple Nobel prizes along the way.\cite{santra_concepts_2009} Fundamental scientific discoveries in early x-ray studies include: illuminating crystal structure by Bragg's x-ray diffraction experiments in 1913,\cite{bragg_reflection_1913} revealing electron orbital state through inelastic x-ray scattering by Compton in 1923,\cite{compton_spectrum_1923} and Klein's generalizations of Compton scattering to higher-energy photons using quantum electrodynamics in 1928.\cite{klein_scattering_1928} In addition to serving as a probe for fundamental investigations of matter, x-rays also find applications in quantifying and identifying industrial pollutants;\cite{luo_determination_2012} imaging oncological and vascular tissues;\cite{butler_bio-medical_2008} and recovering, restoring, or establishing the provenance of works of artistic \cite{janssens_photon-based_2010} or historical \cite{bergmann_archimedes_2007} importance. All these applications are made possible by determining the elemental makeup of the object of interest. \pl

%\begin{figure}
%\includegraphics[width=3in]{../../Figures/Figure0.pdf}% Here is how to import EPS art
%\pcaption{\noindent\fLabel{Motivation}\pStartF An example subsurface reconstruction of a self-portrait, 200mm$\times$300mm, of Australian painter Sir Arthur Streeton (1867-1943) from \citet{howard_high-definition_2012} [a] The visible image of the painting is obscured by primer. [b] A log-scaled intensity image of the obscured painting's zinc content. \pEndF }
%\end{figure}

\subsubsection{\sLabel{Beer}XRF depends on wavelength and element}

\firstp XRF is characterized by absorption of a photon by an electron bound to an atom, followed by `relaxation' of the excited atom by emission of an x-ray photon.\cite{jitschin_progress_1990} Relaxation occurs by XRF when the excited electron is ejected and a higher-shell electron `decays' to fill the vacancy, emitting radiation as a result of the decay (\fRef{Basics}). The relaxation process can consist of a complicated series of excitations and decays,\cite{santra_concepts_2009} but the highest-energy and most-radiative initial excitation and decay is the primary focus of XRF experiments and this paper. Photon absorption and electron ejection happen at well-defined incident photon energies, called the \emph{edge energies}, which are a function of the atomic number of the absorbing atom and the ejected electron's shell.\cite{ueda_high-resolution_2003,de_groot_high-resolution_2001} \fRef{Basics} shows the edge energies and naming conventions associated with the highest-energy and most-dominant\cite{chantler_detailed_2000} absorption levels for the first 90 elements. The tightest-bound, 1s$_{\frac{1}{2}}$ electrons, known as the K-shell electrons, are the dominant contributors to XRF by absorption of the incident photons.\cite{chantler_detailed_2000}

\begin{table}
\begin{ruledtabular}
\begin{tabular}{ l | l  }
  Cross Section &  Formula  \e
  $\sigma_{\text{inelastic}}$ & $\int_{-1}^{1}S(q,Z) \times$ \\ &
  $\{ 2 \pi \dfrac{r_e^2}{2} \dfrac{1+\cos^2{\theta}+\dfrac{E_i (1-\cos\theta)^2}{1+E_i(1-\cos\theta)}}{1+E_i(1-\cos\theta)^2}  d(\cos\theta) \}$ \e
  $\sigma_{\text{elastic}}$ & $ \int_{-1}^{1} |F[q,Z]|^2 \{ 2 \pi \dfrac{r_e^2}{2} (1 + \cos(\theta)^2)  d\cos\theta \}$ \e
  $\sigma_{\text{photoelectric}}$ & $\dfrac{2 h c r_e}{E_i} \text{Im}(F(q,Z))$  
\end{tabular}
\end{ruledtabular}
\pcaption{\tLabel{Scattering}\pStartF Equations for the cross sections relevant to energies typical of XRF. The following conventions are used: q=$\frac{|E_f-E_i|}{c}$ is the momentum transfer, $r_e\equiv\frac{q_e^2}{m_e c^2}$ is the classical electron radius given electron charge $q_e$, speed of light c, and mass m$_e$; $E_i$ and $E_f$ are the incident and final photon energies in units of $m_e c^2$; S(q,Z) is the incoherent scattering function of momentum transfer q and atomic number Z;\cite{hubbell_pair_1980} F(q,Z) is the atomic form factor.\cite{waasmaier_new_1995,hubbell_atomic_1975}}
\end{table}

%\newpage


\section{References}


\section{Bibliography}

\bibliographystyle{unsrtnat}

\bibliography{Paintings} 

\section{Suporting Material}


\end{document}

%
% XXX difference wrt to solids. Auger, no space? \\

%

%
% ****** End of file aipsamp.tex ******
